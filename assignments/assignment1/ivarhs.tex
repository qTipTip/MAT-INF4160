\documentclass{amsart}

\usepackage[]{mathtools} 
\usepackage[]{cleveref} 

\title{Assignment 1 --- MAT-INF4160}
\author{Ivar Haugal{\o}kken Stangeby}
\date{\today}

\begin{document}
\maketitle

\subsection*{Problem 1} 
\label{par:problem_1}

Recall that a \emph{Bernstein polynomial} is a polynomial of the form
\begin{equation}
	\notag
	B_{i}^{n}(x) = {n \choose i} x^i (1 - x)^{n-1}
\end{equation}
where we typically let $x$ live in the unit interval $\left[ 0, 1 \right]$.
We define $n$ to be \emph{degree} of the polynomial, and we call $B_i^n$ the
\emph{$i$'th Bernstein polynomial} of degree $n$ where $i = 0, \ldots, n$.
Note that by the binomial theorem the $n + 1$ Bernstein polynomials of
degree $n$ sum to $1$ on the unit interval.

By application of the product rule and substitution rule we can
differentiate $B_i^n(x)$ with respect to $x$ which yields
\begin{equation}
	\notag
	\frac{d}{dx}B_i^n(x) = n \left(B_{i-1}^{n-1}(x) - B_{i}^{n-1}(x) \right)
\end{equation}
We seek to show that $B_i^n(x)$ has a unique maximum in the interval $[0,
1]$ with this maximum being $x = i / n$. Note that by the extreme value
theorem, since $B_i^n$ is continuous and $[0, 1]$ is compact, we know that
$B_i^n$ does indeed attain a maximum in $[0, 1]$ but it is not neccessarily
unique.

In order to find the maximum, we look at where the derivative is zero.
Consider the derivative of $B_i^n$ where $n \neq 0$. It then suffices to
look at the difference between lower-degree Bernstein polynomials:
\begin{align*}
	{n-1 \choose i-1}x^{i-1}(1 - x)^{n-i} - {n-1 \choose i}x^i(1-x)^{n-i-1} &= 0
\end{align*}
assuming $x \neq 0$ and $x \neq 1$ and we can cancel
common terms which yields
\begin{align*}
	\frac{1}{n-1} - x \left( \frac{1}{n - i} + \frac{1}{i} \right) &= 0
\end{align*}
where solving for $x$ gives us $x = i / n$. We have now shown that under
the condition that $x \neq 0$ and $x \neq 1$ the statement holds. Plugging
in $x = 0$ and $x = 1$ verifies the claim for those two numbers as well.

\subsection*{Problem 2}
\label{sub:problem_2}

We define the \emph{Bernstein approximation} of a function $f: [0, 1] \to
\mathbb{R}$ of order $n$ as the polynomial
\begin{align*}
	g(x) = \sum^{n}_{i=0} f \left( i / n \right) B_i^n(x).
\end{align*}
We want to show that if $f$ is a polynomial of degree $m \leq n$, then $g$
is also a polynomial of degree $m$. Note that differentiating a function
of degree $m$, $m + 1$ times, then you are left with zero. So, the
question we need to ask is whether $g^{(m+1)}(x) = 0$. Make note of the
fact that $g(x)$ is a form of B\'ezier curve where the control points are
given as $c_i = f(i / n)$. We therefore, for convenience sake, represent
$g$ as
\begin{equation}
	\notag
	g(x) = \sum^{n}_{i=0} c_i B_i^n(x).
\end{equation}
We have a formula for the $(m+1)$'th derivative of $g$ on the interval
$[0, 1]$, namely
\begin{equation}
	\label{eq:g_m_diff}
	g^{(m+1)}(x) = \frac{n!}{(n - (m + 1))!} \sum^{\mathclap{n - (m+1)}}_{i=0} \Delta^{m+1}c_iB_i^{n-(m+1)}(x)
\end{equation}
where $\Delta^{k}c_i = \Delta^{k-1}c_{i+1} - \Delta^{k-1}c_{i}$ is the
$k$'th forward difference of $c_i$. We can find a closed form expression for $\Delta c_i$ which is given by
\begin{align*}
	\Delta c_i = f\left(\frac{i + 1}{n}\right) - f \left( \frac{i}{n} \right) = \sum^{m}_{j = 1} a_j \frac{(i+1)^{j-1}}{n^j}, 
\end{align*}
obtained by expanding the expression and using the fact that $(i + 1)^j -
i^j = (i + 1)^{j-1}$. This can be extended to $\Delta^kc_i$ by applying the same tricks. This yields a closed form
\begin{equation}
	\notag
	\Delta^kc_i = \sum^{m}_{j=k} a_j\frac{(i + k)^{j - k}}{n^j}
\end{equation}
it here follows that when $k = m + 1$ we have $\Delta^k = 0$, hence
\cref{eq:g_m_diff} is a sum over zero. Hence $g(x)$ when differentiated $m
+ 1$ times is equal to zero, which means that $g(x)$ is a polynomial of
degree $m$.
\end{document}
